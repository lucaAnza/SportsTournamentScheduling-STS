\documentclass{article}

% Core math + hyperlinks (does not change fonts/margins)
\usepackage{amsmath, amssymb}
\usepackage{hyperref}
\usepackage{graphicx}
\usepackage{booktabs}

% ---- Metadata (edit) ----
\title{CMDO 2025 - Sports Tournament Scheduling
(STS) problem}
\author{%
Luca Anzaldi \texttt{luca.anzaldi@studio.unibo.it} \and
Name Surname \texttt{name.surname@studio.unibo.it}
}
\date{\today}

% ---- Helpful macros (optional) ----
\newcommand{\N}{\mathbb{N}}
\newcommand{\Z}{\mathbb{Z}}
\newcommand{\R}{\mathbb{R}}
\newcommand{\opt}{\mathrm{opt}}
\newcommand{\AllDiff}{\texttt{allDifferent}}

\begin{document}
\maketitle


\section{Introduction}
Briefly introduce the report (not the problem nor the generic methods).
Describe and formalize what is common to all models:
inputs, symbols, shared objective variable and bounds, global pre-processing,
and any assumptions.
Refer back to this section from later sections to avoid repetition.

% Example structure:
\paragraph{Input.}
Define the instance set, indices, and parameters (domains and meanings).

\paragraph{Shared objective variable.}
Define $z \in \R$ (or $\Z$), its bounds, and interpretation.

\paragraph{Pre-processing.}
Summarize any reductions, dominance rules, or canonicalizations applied to all models.

% ---------------- CP ----------------
\section{CP Model}
This section is mandatory.

\subsection{Decision variables}
List all CP variables, with domains and semantics.
For example: $B_i \in \{0,\dots,100\}$ meaning $B_i = j$ iff baker $i$ cooks $j$ cakes.

\subsection{Objective function}
If the objective variable/bounds differ from Section~1, state it here.
Then give the CP objective. Example: minimize total area
$ \sum_{i=1}^{n} W_i \cdot H_i $.

\subsection{Constraints}
State and explain main problem constraints first; then (optionally) implied and symmetry-breaking constraints.

\paragraph{Main constraints.}
Formalize each constraint compactly and explain its role.

\paragraph{Implied constraints (optional).}
Explain why implied constraints hold and how they tighten propagation.

\paragraph{Symmetry breaking (optional).}
Identify model symmetries and show how your constraints reduce them.

\subsection{Validation}
Implement the model in \texttt{MiniZinc}; run at least \texttt{Gecode}
(and optionally other solvers). Design reproducible experiments.

\paragraph{Experimental design.}
Specify solvers, search strategies, hardware/software, time limits, and any per-instance settings.

\paragraph{Experimental results.}
Report objective values (best found or proven optimal).
Mark proven optima in \textbf{bold}. Use \texttt{UNSAT} if proven infeasible; use \texttt{N/A} or ``--'' on timeout.

\begin{table}[h]
\centering
\caption{Results with/without symmetry breaking (example layout).}
\label{tab:cp}
\begin{tabular}{lcccc}
\toprule
\textbf{ID} & \textbf{Chuffed+SB} & \textbf{Chuffed--SB} & \textbf{Gecode+SB} & \textbf{Gecode--SB}\\
\midrule
1 & \textbf{80} & 120 & 80 & 80 \\
2 & 50 & 60 & N/A & N/A \\
3 & UNSAT & UNSAT & N/A & N/A \\
\bottomrule
\end{tabular}
\end{table}

% ---------------- SAT ----------------
\section{SAT Model}
Mandatory for groups of 4; optional (alternative to SMT) for \(\le 3\) students.

\subsection{Decision variables}
Define Boolean literals and their semantics (e.g., $\Delta_{i,j}$ true iff arc $i\!\to\! j$ is used).

\subsection{Objective function}
Explain the optimization strategy in SAT (e.g., linear/binary search over $z$, PB encodings, cardinalities).

\subsection{Constraints}
Describe clause encodings for each modeling component.
State the main constraints first, then implied and symmetry-breaking encodings.

\subsection{Validation}
Implement with at least \texttt{Z3}. If using a solver-independent format (e.g., \texttt{DIMACS}), note how you swap SAT solvers.
Follow the same experimental design/results protocol as in CP.

% ---------------- SMT ----------------
\section{SMT Model}
Mandatory for groups of 4; optional (alternative to SAT) for \(\le 3\) students.

\subsection{Decision variables}
List variables, their sorts, and the theory (e.g., \texttt{Int}, \texttt{Real}, \texttt{UF}, \texttt{BV}).

\subsection{Objective function}
State the objective consistent with Section~1 or differences here.

\subsection{Constraints}
Provide the SMT formulation (theory-specific), with main, implied, and symmetry-breaking parts.

\subsection{Validation}
Use \texttt{Z3} or \texttt{CVC5}. Solver-independent inputs (e.g., \texttt{SMT-LIB}) are encouraged.
Report results as in CP.

% ---------------- MIP ----------------
\section{MIP Model}
This section is mandatory.

\subsection{Decision variables}
List variables and domains (continuous/integer/binary).

\subsection{Objective function}
Linear objective consistent with Section~1 (state differences if any).

\subsection{Constraints}
All constraints must be linear; give compact matrix/inequality forms where possible.

\subsection{Validation}
Run on at least one MIP solver; solver-independent modeling (e.g., \texttt{AMPL}) is a plus.
Report as in CP.

% --------------- Conclusions ---------------
\section{Conclusions}
Summarize findings, modeling trade-offs, and comparative performance insights.

% --------------- Authorship ---------------
\section*{Authenticity and Author Contribution Statement}
We declare that the work reported here is our own and properly cites all external ideas and sources.
Parts of this project (text/code/experiments) were assisted by AI tools where indicated; we disclose the tools used and the exact sections where AI assistance was applied.
\\[0.5em]
\textbf{Author contributions.} Briefly outline the role of each author (modeling, implementation, experiments, analysis, writing).

% --------------- References ---------------
\begin{thebibliography}{9}

\bibitem{minizinc}
MiniZinc: The Modeling Language,
\url{https://www.minizinc.org/}.

\bibitem{z3}
de Moura, Leonardo, and Nikolaj Bj{\o}rner.
Z3: An Efficient SMT Solver.
\emph{TACAS} (2008).

\bibitem{cvc5}
Barbosa et al.
cvc5: A Versatile and Industrial-Strength SMT Solver.
\emph{CAV} (2022).

\end{thebibliography}

\end{document}

